% Thesis 

\chapter[Conclusion]{Conclusion}
\label{chap-conc}

In this thesis, three different robotic manipulators have been described. The first being a liquid nitrogen cooled magnetic manipulator which currently allows for the control of 3 DOF of a cylindrical permanent magnet. This system was able to successfully increase the power output of EM coils using LN2. The second manipulator, similar to the first, was capable of controlling 2 DOF of a pair of spherical magnets. In it, 2 magnetic spheres representing a swarm of robots were used to successfully implement algorithms  for navigation using global inputs and non-slip wall contacts. The last system considered was an augmented 5-DOF robotic arm oriented towards introducing students to the world of robotics and automation. The attachments added feedback to the system, the supporting GUI allowed for automation of the arm's movements,  and allowed inexperienced programmers to enjoy the thrill of programming a robot. 

For future work, the algorithm for the LN2 cooled system will be tweaked to be more efficient and accurate in its control of the permanent magnet. The 2-D manipulation system could be augmented with conical iron cores to further concentrate the magnetic field in the workspace. The vision system can also be upgraded to allow for faster tracking and automating of the navigation process. Finally, for the augmented robotic arm, the custom PCB for interfacing the robot with the drivers and microcontroller can be revised to increase ease of use. The GUI can also be edited to include tabs for image processing, and inverse kinematic motion planing. 

%Here I'll talk breifly about the achievements in each project and what future improvements are to be made.

%For Julien's Project 

%For Shiva's Project, the main goal going forward is automation and finding out how the system reacts to cavities of various shapes, and if the length of the villi wrt to the robotic spheres has any impact on the success of each experiment

%For the arm, I'll concentrate on updating the GUI to include image recognition in its programming and inverse kinematic controls. I'll also talk about including a segment for advanced sensors and maybe image processing. 