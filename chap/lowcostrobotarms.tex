\chapter[Low-Cost Robot Manipulator Arms]{Low-Cost Robot Manipulator Arms}\label{chap-lowcostarms}

The course Introduction to robots, focuses on robotic manipulators.  As a TA for this class, I designed a series of labs using a cheap (under US \$50) robot arm kit. 


\begin{figure}
\centering
{\begin{overpic}[width =\figwid]{Robotarm.jpg}\end{overpic}
}
\caption{\label{fig:defaultarm}{The unmodified Robotic arm}
}
\end{figure}

The kit, as purchased, is not a robot.  It is instead a device with five motors and the user has a small controller that can turn motors on and off. The main aim of the labs are to introduce the students to robotics concepts and control system. A GUI was also implemented for lower level students to introduce them to programming and cultivate interest in robotics. 

My labs are arranged as follows:

\begin{enumerate}
\item  Lab 1:  Manual control
\item  Lab 2:  Open-loop control
\item  Lab 3:  Closed-loop control with Potentiometer sensors
\item  Lab 4:  Closed-loop control with image processing
\end{enumerate}



\section{Lab 1: Open loop control}

For this lab, the students were tasked with assembling the robot and testing it with its default controller. Students are tasked with 
%describe the lab, explain what worked well


\section{Lab 2: Open loop control}

For this lab, the students were tasked with altering the robot to include a microprocessor for open loop control

%describe the lab, explain what worked well

\section{Lab 3: Closed-loop control with Potentiometer sensors}

For this lab, the students were tasked with altering the robot further to include potentiometers to read its joint values and thus implement closed loop control with forward kinematics
%describe the lab, explain what worked well

\section{Lab  4: Closed-loop control with image processing}

For this lab, the students were tasked using camera inputs to control the robot in a simple tracking task. This illustrates closed loop control with inverse kinematics. 

%describe the lab, explain what worked well

\section{Pre-semester prep}

%What materials need to be ordered?  What needs to be set up?


\section{Results}

%explain the outcome of the labs