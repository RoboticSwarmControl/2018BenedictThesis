\chapter[Low-Cost Robot Manipulator Arms]{Low-Cost Robot Manipulator Arms}\label{chap-lowcostarms}

The course Introduction to robots focuses on robotic manipulators.  As a TA for this class, I designed a series of labs using a cheap (under US \$50) robot arm kit. 


\begin{figure}
\centering
{\begin{overpic}[width =\figwid]{Robotarm.jpg}\end{overpic}
}
\caption{\label{fig:defaultarm}{The unmodified Robotic arm}
}
\end{figure}

The kit, as purchased, is not a robot.  It is instead a device with five user controlled motors that can be turned on or off. The main aim of these labs are to introduce the students to robotics concepts and control systems. A GUI was also implemented for lower level students to introduce them to programming a simple robot and cultivate interest in robotics. 

My labs are arranged as follows:

\begin{enumerate}
\item  Lab 1:  Assembly and familiarization
\item  Lab 2:  Open-loop control
\item  Lab 3:  Closed-loop control with Potentiometer sensors
\item  Lab 4:  Closed-loop control with image processing
\end{enumerate}



\section{Lab 1: Assembly and familiarization}

For this lab, the students were tasked with assembling the robot and controlling it with its default controller. They were tasked with moving around little objects within the robot's workspace to get an idea of the robot's capabilities. By assembling the robot themselves, the students become intimately familiar with the limits of the robot, and its possible orientations. 
%describe the lab, explain what worked well


\begin{figure}
\centering
{\begin{overpic}[width =\figwid]{Preassembly.jpg}\end{overpic}
}
\caption{\label{fig:preassembly}{The parts of the robot arm before assembly}
}
\end{figure}

\begin{figure}
\centering
{\begin{overpic}[width =\figwid]{Assembled1.jpg}\end{overpic}
}
{\begin{overpic}[width =\figwid]{Assembled2.jpg}\end{overpic}
}
\caption{\label{fig:Assembly1}{The Assembled Robotic arm}
}
\end{figure}

\section{Lab 2: Open loop control}

For this lab, the students were tasked with altering the robot to include a microprocessor for open loop control. Using a motor driver and an arduino microprocessor, the students were instructed on how to include a degree of automation to the robot's movements. The main aim of this project was to showcase the ease to which automation can be achieved, but also make them aware of how unreliable automation is without sensor feedback for control. 

%describe the lab, explain what worked well

\section{Lab 3: Closed-loop control with Potentiometer sensors}

For this lab, the students were tasked with altering the robot further to include potentiometers to read its joint values and thus implement closed loop control with forward kinematics. The students were provided with instructions on how to attach the potentiometers to the robot with some custom designed 3D printed parts and lasercut gears. The aim of this lab was to help students understand the importance of implementing feedback in robotics control systems, and to  a certain extent, introduce them to fabrication and design. 


\begin{figure}
\centering
{\begin{overpic}[width =\figwid]{Augment1.jpg}\end{overpic}
}
{\begin{overpic}[width =\figwid]{Augment2.jpg}\end{overpic}
}
{\begin{overpic}[width =\figwid]{Augment3.jpg}\end{overpic}
}
\caption{\label{fig:Assembly1}{The Augmented Robotic arm}
}
\end{figure}
%describe the lab, explain what worked well

\section{Lab  4: Closed-loop control with image processing}

For this lab, the students were tasked with using camera inputs to control the robot in a simple tracking task. This illustrates closed loop control with inverse kinematics. With this lab the students are introduced to another method of sensory feedback, and also given an  introduction to image processing.


%describe the lab, explain what worked well

\section{Pre-semester prep}
 
Before the semester begins, the acrylic gears are cut, the 3D printed parts are made, the jumper cables, the custom PCB circuit board and a few other miscellaneous parts for the augmentation are ordered.
%What materials need to be ordered?  What needs to be set up?


\section{Results}

At the end of the labs, the students leave the class with experience constructing and programming a robot arm. In the course of doing this, they learn robotics terminology, and have an idea about different means of controling robots and implementing sensor feedback. 
%explain the outcome of the labs